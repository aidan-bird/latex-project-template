%
% Aidan Bird, 2021
%
% template doc
%

\documentclass[10pt,a4paper]{article}
\newcommand{\projectroot}{/XXX}
% Aidan Bird, 2021
%
% latex doc header
%
\usepackage{mathtools}
\usepackage{hyperref}
\usepackage{graphicx}
\usepackage{color}
\usepackage{float}
\usepackage{adjustbox}
\usepackage{datetime}
\usepackage{placeins}
\newdateformat{autodate}{\THEDAY~\monthname[\THEMONTH] \THEYEAR}
\usepackage{booktabs}

\begin{document}
\title{XXX}
\author{Aidan Bird}
\date{\autodate\today}
\maketitle
\phantomsection
\addcontentsline{toc}{section}{Introduction}

% pol60
% \begin{figure}[H]
% \centering
% \begin{adjustbox}{width=\textwidth}
% \includegraphics{\projectroot/src/pol60.pdf}
% \end{adjustbox}
% \end{figure}

\newpage

\section*{Objective} 

\section*{Analysis} 

\subsection*{1)}

% FIGURE EXAMPLE
% \begin{figure}[H]
%     \centering
%     \begin{adjustbox}{width=\textwidth}
%         \input{\projectroot/figures/fig1.tex}
%     \end{adjustbox}
%     \caption{The graph of the I-V curves at varying gate voltages. The black
%     portions of the curves are the linear regions and the black circles
%     indicate where the saturation region begins.}
%     \label{figure1}
% \end{figure}

\FloatBarrier

% TABLE EXAMPLE
% \begin{figure}[H]
%     \centering
%     \begin{tabular}{llr}  
%         \toprule
%         Gate Voltage (V) & Voltage (V) & Current ($\mu$A) \\
%         \midrule
%         1 & 0.2 & 4  \\
%         2 & 0.6 & 20 \\
%         3 & 0.7 & 36 \\
%         4 & 0.8 & 56  \\
%         5 & 0.9 & 79   \\
%         \bottomrule
%     \end{tabular}
%     \caption{current-voltage characteristics of the end of the linear region
%     (end of the black lines in Figure \ref{figure1}) for various gate voltages.
%     }
%     \label{figure2}
% \end{figure}

\FloatBarrier

\newpage

\subsection*{2)}

% MATH EXAMPLE
% \begin{align*}
%     I_D &= K_n [ 2 (V_{GS} - V_{TN}) V_{DS} ] \\
%     &= 2 K_n V_{DS} V_{GS} - 2 K_n V_{DS} V_{TH} \\
%     \text{The slope } M &= 2 K_n V_{DS} \\
%     K_n &= \frac {M} {2 V_{DS}} \\
% \end{align*}

\section*{Appendix} 

\end{document}
